\documentclass[conference]{IEEEtran}
\IEEEoverridecommandlockouts
% The preceding line is only needed to identify funding in the first footnote. If that is unneeded, please comment it out.
\usepackage{cite}
\usepackage{amsmath,amssymb,amsfonts}
\usepackage{algorithmic}
\usepackage{graphicx}
\usepackage{textcomp}
\usepackage{xcolor}
\def\BibTeX{{\rm B\kern-.05em{\sc i\kern-.025em b}\kern-.08em
    T\kern-.1667em\lower.7ex\hbox{E}\kern-.125emX}}
\renewcommand\IEEEkeywordsname{Keywords}

\begin{document}

\title{ Implementation of Grover’s Algorithm based on
	Quantum Reservoir Computing  \\

}
\author{\IEEEauthorblockN{Shivani Mehta}
	\IEEEauthorblockA{\textit{Department of ECE,} \\
		\textit{IIITDM Kancheepuram,}\\
		Chennai-600127, India.\\
		ec22m2002@iiitdm.ac.in}
	\and
	\IEEEauthorblockN{Sajja Jyothikrishna}
	\IEEEauthorblockA{\textit{Department of ECE,} \\
		\textit{IIITDM Kancheepuram,}\\
		Chennai-600127, India. \\
		ec21b1022@iiitdm.ac.in}
	\and
	\IEEEauthorblockN{V.Praveen Bhallamudi}
	\IEEEauthorblockA{\textit{Department of Physics,} \\
		\textit{IIITDM Kancheepuram,}\\
		Chennai-600036, India. \\
		praveen.bhallamudi@iitm.ac.in}
	\and
	\IEEEauthorblockN{Sumanth Arige}
	\IEEEauthorblockA{\textit{Department of ECE,} \\
		\textit{IIITDM Kancheepuram,}\\
		Chennai-600127, India. \\
		edm20d010@iiitdm.ac.in}
	\and

	\IEEEauthorblockN{ Tejendra Dixit, $Member, IEEE$}
	\IEEEauthorblockA{\textit{Department of ECE,} \\
		\textit{IIITDM Kancheepuram,}\\
		Chennai-600127, India. \\
		tdixit@iiitdm.ac.in}
}

\maketitle

\begin{abstract}
	Quantum computing represents the leading edge of
	computational technology, leveraging the principles of quantum
	mechanics to execute targeted computations much faster than
	classical computers. In contrast to classical bits, which are limited
	to representing either 0 or 1, qubits, or quantum bits, exhibit
	the extraordinary property of superposition. This distinctive
	characteristic enables qubits to simultaneously occupy multiple
	states, empowering quantum computers to explore numerous
	potential solutions to a problem concurrently. This feature
	makes quantum computing particularly potent for specific tasks.
	Recent research endeavors have been sparked by the potential
	of advanced quantum computing technology, leading to the
	creation of simulations of quantum computers using classical
	hardware. Grover’s quantum search algorithm serves as a notable
	illustration of quantum computing application, enabling quantum
	computers to conduct a database search within an unsorted array
	with a quadratic speedup in time efficiency compared to classical
	computers. This document presents the quantum Grover search
	algorithm and its application through 5-qubit quantum circuits,
	as well as a design framework to simplify the creation of an
	oracle for a greater number of qubits.
\end{abstract}

\begin{IEEEkeywords}
	Quantum computation, Qubits, Oracle, Grover’s
	algorithm, IBM Qiskit.
\end{IEEEkeywords}

\section{Introduction}
The exploration of quantum computing [1][2] falls within
the realm of quantum information science, which revolves
around the fundamental principles of storing and manipulating
information. In this work, we delved into quantum computing,
acquiring a comprehensive understanding of quantum bits
and their properties, as well as leveraging these properties
to tackle problems. We familiarized ourselves with quantum
gates and their operations on qubits, simulating all the funda
mental quantum gates [3][4]. Additionally, we delved into the
Grover search algorithm and implemented quantum gates for
Grover operations. Quantum computers exhibit significantly
faster speeds compared to classical computers [5][6]. In the
case of an unsorted dataset with size N, classical computers
usually demand O(N) operations, whereas Grover’s algorithm
accomplishes this task optimally in O($\sqrt{N}$) operations.
\\
We
executed the algorithm using Qiskit, an IBM tool for com
puting quantum circuits, and conducted simulations for the
Grover algorithm [7], presenting the results graphically with
the probability of obtaining the correct output.

Within Grover’s quantum search algorithm, a network with
n qubits harbors $ 2^{n} $ = N states, with each state bearing a
probability of $ 1/N $ for discovery. Consequently, the amplitude
of each state is $ 1/\sqrt{N} $
. Conversely, classical systems tackling the
same problem necessitate a maximum of O(N) trials.
\section{Results and Discussions}
\subsection{ Models Used}\label{AA}

To precisely replicate the switching behavior of the Mott material (V\textsubscript{2}O\textsubscript{3}), an extensive model has been developed using established literature [15-18]. Three distinct models were devised, featuring various electrode shapes, with V\textsubscript{2}O\textsubscript{3} as the switching material and an Al\textsubscript{2}O\textsubscript{3} substrate with Ti electrodes in each model. A controlled current is applied to one electrode, generating heat within the device and triggering the Insulator-to-Metal transition. The dimensions of all models are on a micrometer scale. To scrutinize the transition dynamics of V\textsubscript{2}O\textsubscript{3} in each model, temperature and electric potential profiles were meticulously recorded.We believe that the RS properties and the electric field distribution may be affected in some way by the geometry of the electrode, similar to the form of the corners in the lateral devices. We produced three distinct electrode forms in order to investigate the impact of electrode shape that are
: rectangle, cylindrical, and triangle electrode . The models utilized is depicted in \textbf{Fig. 1, 3, 5}
% \begin{figure}[htbp]
% 	\centerline{\includegraphics[width=9cm,height=9cm,keepaspectratio]{fig1.jpg}}
% 	\caption{Rectangular electrodes model.}
% 	\label{fig}
% \end{figure}



\subsection{Setup using Rectangular Electrodes}

The observation of potential and temperature changes in material for rectangular electrodes is depicted in \textbf{Figs. 2(a-d)}. The electrode gaps utilized in the setup depicted in Fig. 2 measure 100 nm. Figs.  \textbf{2(a, c)} depicts the precise state of the model at the moment of IMT occurrence, while \textbf{Figs. 2(b, d)} showcases the subsequent state following IMT and its attainment of equilibrium. Upon careful observation, it becomes evident that the material changes into a conductive state which can be observed from the changes in electric potential and temperature across the device. These results are more pronounced due to wider electrodes and more area allowing better heat transfer to substrate.Based on the provided diagram, it is evident that the IMT transition takes place precisely at a temperature of 150 K validating existing literature.

% \begin{figure}[htbp]
% 	\centerline{\includegraphics[width=9cm,height=10cm,keepaspectratio]{fig2.jpg}}
% 	\caption{Temperature and potential  of the 100 nm gap setup taken at different intervals for rectangular electrodes.}
% 	\label{fig}
% \end{figure}

\subsection{Setup using Triangular Electrodes}

% \begin{figure}[htbp]	\centerline{\includegraphics[width=8cm,height=8cm,keepaspectratio]{fig3.jpg}}
% 	\caption{Triangular electrodes model}
% 	\label{fig}
% \end{figure}
The potential and temperature changes in material for triangular electrodes is observed here. The geometric sharp corners of the electrodes provide the strongest electric field, as seen in \textbf{ Fig. 4}. The material's ability to resist switching is aided by the triangle electrode's ability to create strong electric fields close to the metal/oxide contact.
This characteristic enhances the material's capacity for Resistive Switching, contributing to its overall effectiveness in the process.

% \begin{figure}[htbp]
% 	\centerline{\includegraphics[width=9cm,height=10cm,keepaspectratio]{fig4.jpg}}
% 	\caption{Temperature and potential profile  of the 100 nm gap setup taken at  different intervals for triangular electrodes.}
% 	\label{fig}
% \end{figure}



\subsection{Setup using Cylindrical Electrodes}


% \begin{figure}[htbp]
% 	\centerline{\includegraphics[width=9cm,height=10cm,keepaspectratio]{fig5.jpg}}
% 	\caption{ Cylindrical electrodes model.}
% 	\label{fig}
% \end{figure}
The cylindrical electrode model differs from the previous two models in its construction. It follows a sandwiched configuration where V\textsubscript{2}O\textsubscript{3} is positioned between two electrodes, with the substrate covering the entire device.The behaviour of the setup has been depicted in \textbf{Fig 6} in which we can we clearly see the formation of conducting filament in the material V\textsubscript{2}O\textsubscript{3}. Building this setup is more expensive compared to the previous ones because those setups can take advantage of existing silicon-based fabrication methods. This particular configuration is designed primarily for academic research, as it provides complete isolation of the device through the substrate. The study focuses on observing potential and temperature changes within the material using cylindrical electrodes. In this setup, the electrode gaps are set at 100 nm, and the comparison is carried out at various intervals.These results are similar to what we observed in previous electrodes but the conducting filament is clearly observed due to the sandwiched setup.

% \begin{figure}[htbp]
% 	\centerline{\includegraphics[width=9cm,height=10cm,keepaspectratio]{fig6.jpg}}
% 	\caption{   Temperature and potential profile  of the 100 nm gap setup taken at  different intervals for Cylindrical electrodes.}
% 	\label{fig}
% \end{figure}



\subsection{Comparison of three models}

The comparison of results from all three models with a 100 nm gap aims to determine the most efficient electrode shape. Upon comparing the voltage-current (V-I) characteristics and voltage-versus-time profiles of all three models, notable insights emerge. The switching point voltage, observed at the peaks of each curve, varies among the setups. Specifically, the triangular electrode configuration exhibits the highest switching point voltage, while the rectangular setup shows the lowest. This implies that the rectangular setup boasts superior switching capabilities at lower voltages, contributing to reduced power consumption. Analyzing the resistance-versus-time curve, the triangular setup displays a significantly steeper curve, indicating faster resistance switching compared to the other models. This accelerated switching is attributed to the focusing effect inherent in the triangular configuration.



% \begin{figure}[htbp]
% 	\centerline{\includegraphics[width=8cm,height=8cm,keepaspectratio]{fig7.jpg}}
% 	\caption{Comparison of  V-I characteristics of  Rectangular, Triangular and Cylindrical electrode setups of 100 nm gap. }
% 	\label{fig}
% \end{figure}

Remarkably, the resistance switching in the rectangular electrode setup closely mirrors that of the triangular configuration. In contrast, the cylindrical setup exhibits a comparatively slower switching behavior.We evaluate resistances, V-I characteristics, and potential differences. Interestingly, the rectangular electrode outperforms the others, with lower resistance and a lower activation voltage. This superiority can be attributed to the rectangular electrode’s capacity to efficiently transfer heat to the material, thanks to its larger surface area. This outcome might appear counterintuitive, as one might expect the triangular electrode to perform better due to its focusing effect. However, the smaller contact area of the triangular electrode limits the heat transfer compared to the rectangular electrode.
% \begin{figure}[htbp]
% 	\centerline{\includegraphics[width=8cm,height=8cm,keepaspectratio]{fig8.jpg}}
% 	\caption{Comparison of resistances of  Rectangular, Triangular and Cylindrical electrode setups of 100 nm gap. }
% 	\label{fig}
% \end{figure}
% \begin{figure}[htbp]
% 	\centerline{\includegraphics[width=8cm,height=8cm,keepaspectratio]{fig9.jpg}}
% 	\caption{Comparison of Potential differences of  Rectangular, Triangular and Cylindrical electrode setups of 100 nm gap.
% 	 }
% 	\label{fig}
% \end{figure}
The cylindrical electrode exhibits the least favorable performance, primarily due to the difficulty in achieving higher temperatures in this configuration, owing to the larger substrate. This underscores the importance of prioritizing heat accumulation when designing these setups. Furthermore, this observation supports the validity of the Joule heating model in understanding the Insulator-to-Metal Transition (IMT) in Mott devices

\section{Conclusion}
This study thoroughly investigated and modeled the Mott insulator V\textsubscript{2}O\textsubscript{3}, exploring its potential for ReRAM technology. The impact of various electrode configurations on device performance was carefully examined, identifying crucial areas for improvement crucial for successful integration. These results are pivotal for advancing the use of Mott materials in emerging memory technologies. The insights gained from this research are particularly valuable for developing a practical, physical model for future applications.




\begin{thebibliography}{00}

	\bibitem{b1} T. N. Theis and P. M. Solomon, "In Quest of the “Next Switch”: Prospects for Greatly Reduced Power Dissipation in a Successor to the Silicon Field-Effect Transistor," in Proceedings of the IEEE, vol. 98, no. 12, pp. 2005-2014, Dec. 2010.

	\bibitem{b2} Molas, G.; Nowak, E. "Advances in Emerging Memory Technologies: From Data Storage to Artificial Intelligence," Appl. Sci., vol. 11, no. 23, p. 1254, 2021.

	\bibitem{b3} J. H. de Boer, and E. J. W Verwey, "Semi-conductors with partially and with completely filled 3d-lattice bands," Proceedings of the Physical Society, vol. 49 (4S), pp. 4S, Aug.1937.

	\bibitem{b4} E. Janod, J. Tranchant, B. Corraze, M. Querré, P. Stoliar, M. Rozenberg, T. Cren et al. "Resistive switching in Mott insulators and correlated systems," Advanced Functional Materials, vol. 25, no. 40, pp. 6287-6305, Oct.2015.

	\bibitem{b5} J. Hubbard, "Electron correlations in narrow energy bands. II. The degenerate band case." Proceedings of the Royal Society of London. Series A. Mathematical and Physical Sciences, vol. 277, no. 1369, pp. 237-259, Jan.1964.

	\bibitem{b6} C. N. Berglund, "Thermal filaments in vanadium dioxide," in IEEE Transactions on Electron Devices, vol. 16, no. 5, pp. 432-437, May 1969.

	\bibitem{b7} A. Ronchi et al., "Light-Assisted Resistance Collapse in a V\textsubscript{2}O\textsubscript{3}-Based Mott-Insulator Device," Phys. Rev. Appl., vol. 15, no. 4, pp. 044023, Apr. 2021.

	\bibitem{b8} V. N. Andreev et al., "Thermal conductivity of VO\textsubscript{2}, V3O5, and V\textsubscript{2}O\textsubscript{3}," physica status solidi (a), vol. 48, no. 2, pp. K153-K156, 1978.

	\bibitem{b9} P. Stoliar et al., "Universal electric-field-driven resistive transition in narrow-gap Mott insulators," Adv. Mater., vol. 25, pp. 3222, 2013.

	\bibitem{b10} S. Guénon et al., "Electrical breakdown in a V\textsubscript{2}O\textsubscript{3} device at the insulator-to-metal transition," EPL (Europhysics Letters), vol. 101, pp. 57003, 2013.

	\bibitem{b11} P. Homm et al., "Collapse of the low temperature insulating state in Cr-doped V\textsubscript{2}O\textsubscript{3} thin films," Appl. Phys. Lett., vol. 107, pp. 111904, 2015.

	\bibitem{b12} Y. Kalcheim et al., "Non-thermal resistive switching in Mott insulator nanowires," Nat. Commun., vol. 11, pp. 2985, 2020.

	\bibitem{b13} M. M. Qazilbash et al., "Electrodynamics of the vanadium oxides VO\textsubscript{2} and V\textsubscript{2}O\textsubscript{3}," Phys. Rev. B, vol. 77, pp. 115121, 2008.

	\bibitem{b14} P. Paweł, J. Jamroz, and T. K. Pietrzak, “Observation of metal–insulator transition (mit) in vanadium oxides V\textsubscript{2}O\textsubscript{3} and VO\textsubscript{2} in xrd, dsc and dc experiments,” Crystals, 2023.

	\bibitem{b15} P. Homm, M. Menghini, J. W. Seo, S. Peters, and J. P. Locquet, “Room temperature Mott metal–insulator transition in V\textsubscript{2}O\textsubscript{3} compounds induced via strain-engineering,” APL Materials, vol. 9, no. 2, pp. 021116, Feb  2021.

	\bibitem{b16} F. Mazzola, S. K. Chaluvadi, V. Polewczyk, D. Mondal, J. Fujii, P. Rajak, M. Islam, R. Ciancio, L. Barba, M. Fabrizio, G. Rossi, P. Orgiani, and I. Vobornik, “Disentangling structural and electronic properties in V\textsubscript{2}O\textsubscript{3} thin films: A genuine non symmetry breaking mott transition,” Nano Letters, vol. 22, no. 14, pp. 5990–5996, 2022.

	\bibitem{b17} H. Kizuka et al., “Temperature dependence of thermal conductivity of VO\textsubscript{2} thin films across metal–insulator transition,” Japanese Journal of Applied Physics, vol. 98, no. 053201, 2015.

	\bibitem{b18} H. Y. Peng, L. Pu, J. C. Wu, D. Cha, J. H. Hong, W. N. Lin, Y. Y. Li, J. F. Ding, A. David, K. Li, and T. Wu, “Effects of electrode material and configuration on the characteristics of planar resistive switching devices,” APL Materials, vol. 1, no. 5, pp. 052106, Nov  2013.
\end{thebibliography}

\end{document}
